 \textbf{Directions}: Complete the following programming tasks.  Write proper MATLAB code required to complete the task.  You may use MATLAB to develop and test your solutions.  


\begin{outline}[enumerate]
% ----------------------------------------------------- %



% ----- PROBLEM ----- %
\1 How would you get help on MATLAB structures?

\if \sols1
% Solution
\textbf{Solution}: 
{\color{blue}
% MATLAB CODE
\begin{lstlisting}
help struct
\end{lstlisting}
}
\fi

\vfill 


% ----- PROBLEM ----- %
\1\label{pro1} Create an empty structure named \texttt{exam} with one field name, \texttt{n}.


\if \sols1
% Solution
\textbf{Solution}: 
{\color{blue}
% MATLAB CODE
\begin{lstlisting}
exam = struct('n',[]);
\end{lstlisting}
}
\fi

\vfill 


\newpage 



% ----- PROBLEM ----- %
\1 Suppose there (already) exists a text file, {\tt number.txt}, containing a list of positive integers.  Read this file and select a random entry, then store that value in \texttt{n} from Problem \ref{pro1}   


\if \sols1
% Solution
\textbf{Solution}: 
{\color{blue}
% MATLAB CODE
\begin{lstlisting}

fid = fopen('numbers.txt','r');
nums = zeros(10,1);
i = 1;
while ~feof(fid)
    nums(i) = str2double(fgetl(fid));
   % fprintf('%d\n',fgetl(fid));
   i = i +1;
end
fclose(fid);
nums = nums(nums > 0);

% Alt. 
% fid = fopen('numbers.txt','r');
% A = fscanf(fid,'%d',[1,inf])';
% fclose(fid);

% Random
randele = randi([1, length(nums)],1,1);
exam.n = nums(randele);
\end{lstlisting}
}
\fi

\vfill 



% ----- PROBLEM ----- %
\1 Create a random integer vector over the range $-10$ to  $10$ of length \texttt{n} from Problem \ref{pro1}  Store this vector in the the \texttt{exam} structure in a field named {\tt x}.

\if \sols1
% Solution
\textbf{Solution}: 
{\color{blue}
% MATLAB CODE
\begin{lstlisting}
exam.x = randi([-10,10], 1, exam.n);
\end{lstlisting}
}
\fi

\vfill


\newpage  







% ----- PROBLEM ----- %
\1 Determine how many values are negative in the vector {\tt x} of the {\tt exam} structure defined in \ref{pro1} 

\if \sols1
% Solution
\textbf{Solution}: 
{\color{blue}
% MATLAB CODE
\begin{lstlisting}
sum(exam.x < 0);

% Alt.
numneg = 0;
for i = 1:length(exam.x)
   if exam.x(i) < 0
       numneg = numneg +  1;
   end
end
\end{lstlisting}
}
\fi

\vfill 




% ----- PROBLEM ----- %
\1\label{proby} Change every entry of {\tt x} in the {\tt exam} structure to be positive (e.g., $-12 \to 12$).  Store this ``positives'' vector in a field named {\tt y}.

\if \sols1
% Solution
\textbf{Solution}: 
{\color{blue}
% MATLAB CODE
\begin{lstlisting}

exam.y = abs(exam.x);

% alt.
exam.y = exam.x;
for i = 1:length(exam.x)
   if exam.y(i) < 0 
      exam.y(i) = -exam.y(i);
   end
end
\end{lstlisting}
}
\fi

\vfill 




\newpage 




% ----- PROBLEM ----- %
\1 Determine if there are any zeros in the ``positives'' vector {\tt y} of the  structure, see Problem \ref{proby}

\if \sols1
% Solution
\textbf{Solution}: 
{\color{blue}
% MATLAB CODE
\begin{lstlisting}
any(exam.y == 0);

% Alt. 
negs = false;
for i = 1:length(exam.y)
    if exam.y(i) < 0
        negs = true;
        break
    end
end
\end{lstlisting}
}
\fi

\vfill 








% ----- PROBLEM ----- %
\1 Export the list of ``positives'' (i.e.,  {\tt y } from the {\tt exam } structure) to a file named {\tt positives.txt}.  

\if \sols1
% Solution
\textbf{Solution}: 
{\color{blue}
% MATLAB CODE
\begin{lstlisting}
fid = fopen('positives.txt','w');
for i = 1:length(exam.y)
   fprintf(fid,'%d\n',exam.y(i)); 
end
fclose(fid);
\end{lstlisting}
}
\fi

\vfill 


 
\newpage 

 
 

% ----- PROBLEM ----- %
\1 Write code to determine \underline{how many terms} in the sequence $3k^2$, for $k$ equal to the elements of the ``positives'' vector {\tt y} are required for the sum of the terms to exceed $100 \cdot n$ where $n$ is the length of this list.
Store this value in the {\tt exam} structure in a field named {\tt num\_terms}.  

\if \sols1
% Solution
\textbf{Solution}: 
{\color{blue}
% MATLAB CODE
\begin{lstlisting}
% v.1
find(cumsum(3*(exam.y).^2) < 100*length(exam.y), 1,'last');

%% Alternatively
f = @(k) 3*k.^2;
S = 0;
numterms = -1;
while (S < 100*length(exam.y))
   numterms = numterms + 1;
   S = S +  f(exam.y(numterms+1));   
end
fprintf('It takes %d terms to exceed that value.\n\n', numterms);
\end{lstlisting}
}
\fi

\vfill 





% ----- PROBLEM ----- %
\1  Represent the following data on Homer J. Simpson that includes (name, street address, city, ages of everyone in household, dog's name, and beers per day of the week)  using a cell array.  \underline{Note} the apostrophe in the dog's name.  Then specify the code needed to find the age of Homer's kid Bart, who is $10$ years old.   
\begin{table}[htbp]
  \centering
  \begin{tabular}{lll} % column spec
    \toprule
    Homer Simpson & 742 Evergreen Terrence & Springfield\\
    $[42, 39, 10, 8, 3]$ & Santa's Little Helper & $[2,2,4,1,3,6,9]$ \\
    \bottomrule
  \end{tabular}
  \caption{Data on Homer J. Simpson}
  \label{tab:basic-align}
\end{table}



% homer{2,1}(3)
\if \sols1
% Solution
\textbf{Solution}: 
{\color{blue}
% MATLAB CODE
\begin{lstlisting}

homer = {'Homer J. Simpson', '742 Evergreen Terrence', 'Springfield';...
        [42, 39, 10, 8, 3], 'Santa''s Little Helper' ,[2,2,4,1,3,6,9]};
        
homer{2,1}(3);
\end{lstlisting}
}
\fi

\vfill 



\end{outline}